\documentclass[9pt]{developercv} % Default font size, values from 8-12pt are recommended

%----------------------------------------------------------------------------------------

\begin{document}

%----------------------------------------------------------------------------------------
%	TITLE AND CONTACT INFORMATION
%----------------------------------------------------------------------------------------

\begin{minipage}[t]{0.45\textwidth} % 45% of the page width for name
	\vspace{-\baselineskip} % Required for vertically aligning minipages

	% If your name is very short, use just one of the lines below
	% If your name is very long, reduce the font size or make the minipage wider and reduce the others proportionately
	\colorbox{black}{{\HUGE\textcolor{white}{\textbf{\MakeUppercase{Gaetano}}}}} % First name

	\colorbox{black}{{\HUGE\textcolor{white}{\textbf{\MakeUppercase{D'Agostino}}}}} % Last name

	\vspace{6pt}

	{\huge Sviluppatore} % Career or current job title
\end{minipage}
\begin{minipage}[t]{0.275\textwidth} % 27.5% of the page width for the first row of icons
	\vspace{-\baselineskip} % Required for vertically aligning minipages

	% The first parameter is the FontAwesome icon name, the second is the box size and the third is the text
	% Other icons can be found by referring to fontawesome.pdf (supplied with the template) and using the word after \fa in the command for the icon you want
	\icon{MapMarker}{12}{Milan, Italy}\\
	\icon{At}{12}{\href{mailto:dagyu95dev@gmail.com}{dagyu95dev@gmail.com}}\\	
\end{minipage}
\begin{minipage}[t]{0.275\textwidth} % 27.5% of the page width for the second row of icons
	\vspace{-\baselineskip} % Required for vertically aligning minipages

	% The first parameter is the FontAwesome icon name, the second is the box size and the third is the text
	% Other icons can be found by referring to fontawesome.pdf (supplied with the template) and using the word after \fa in the command for the icon you want
	\icon{Globe}{12}{\href{https://dagyu.dev/}{dagyu.dev}}\\
	\icon{Github}{12}{\href{https://github.com/dagyu}{github.com/dagyu}}\\
	\icon{Linkedin}{12}{\href{https://www.linkedin.com/in/dagyu}{in/dagyu}}\\
\end{minipage}

\vspace{0.5cm}

%----------------------------------------------------------------------------------------
%	INTRODUCTION
%----------------------------------------------------------------------------------------


\begin{figure}[!htb]
    \centering
    \begin{minipage}{0.5\textwidth} % 40% of the page width for the introduction text
    \cvsect{Chi Sono?}
    \vspace{0.3cm}
	\vspace{-\baselineskip} % Required for vertically aligning minipages
	
	Nato nel 1995 a Leonforte (EN) e attualmente residente a Milano.\\
	Sono una persona molto curiosa e amante delle sfide.\\

	\end{minipage}
	\hfill % Whitespace between
	\begin{minipage}{0.4\textwidth} % 50% of the page for the skills bar chart
		\centering
		\shadowimage[width=.8\textwidth]{images/profile.jpg}
	\end{minipage}
\end{figure}

\vspace{0.5cm}

%----------------------------------------------------------------------------------------
%	ADDITIONAL INFORMATION
%----------------------------------------------------------------------------------------


% \begin{minipage}[t]{0.3\textwidth}
% 	\vspace{-\baselineskip} % Required for vertically aligning minipages

% 	\cvsect{What I'm looking for?}

% 	I am looking for a stimulating job full of new challenges.
% 	My goal is to be a software engineer so I would like to work in a company where
% 	I can grow professionally.
% \end{minipage}
% \hfill
% \begin{minipage}[t]{0.3\textwidth}
% 	\vspace{-\baselineskip} % Required for vertically aligning minipages

% 	\cvsect{what can i offer?}

% 	I love learning about new technologies,
% 	expanding my horizons and talking to people because I think they are a great source of news.
% 	So in me you will find a person who loves challenges and maintains good harmony within
% 	the group to achieve maximum results.
% \end{minipage}
% \hfill
% \begin{minipage}[t]{0.3\textwidth}
% 	\vspace{-\baselineskip} % Required for vertically aligning minipages
	
% 	\cvsect{Skills and Experiences}
	
% 	My skills and experiences are always growing and evolving so I prefer not to include them here 
% 	but you can find the updated version on my \href{https://dagyu.netlify.app/}{website}.
% \end{minipage}

%----------------------------------------------------------------------------------------
%	EXPERIENCE
%----------------------------------------------------------------------------------------

\cvsect{Esperienze}

\begin{entrylist}
	\entry
	{Mar 2023 -- \\Dic 2023\\\footnotesize{freelance}}
	{Sviluppatore Full stack}
	{\href{https://www.perform-up.com/}{Perform-UP}}
	{
		Sviluppato 3 app mobile per un cliente nel campo del training mentale sportivo. In particolare Perform-UP Tennis, Perform-UP Golf e Perform-UP Volley.
		\\
		\texttt{Dart}\slashsep
		\texttt{[Type|Java]Script}\slashsep
		\texttt{NodeJS}\slashsep
		\texttt{Flutter}\slashsep
	}
\end{entrylist}

\begin{entrylist}
	\entry
	{Mag 2023 -- \\Nov 2023\\\footnotesize{freelance}}
	{Sviluppatore Front-End}
	{\href{https://www.studeogroup.it/}{Cefriel}}
	{
		Ho collaborato con Studeo Group per sviluppare un'app web per un cliente che l'ha utilizzata come portfolio per la presentazione di tutti i prodotti della loro azienda. 
		\\
		\texttt{React}\slashsep
	}
\end{entrylist}

\begin{entrylist}
	\entry
	{Sep 2021 -- \\Sep 2022\\\footnotesize{internship in \href{https://www.cefriel.com/}{Cefriel}}}
	{Sviluppatore Front-End/Sviluppatore DevOps}
	{\href{https://www.cefriel.com/}{Cefriel}}
	{
		Durante questa esperienza ho approfondito le mie conoscenze nel campo front-end, studiando e applicando nuovi pattern di sviluppo ai componenti; in particolare l'Atomic Design.
		\newline Ho anche contribuito all'introduzione di pratiche DevOps nell'azienda.
		\\
		\texttt{Angular}\slashsep
		\texttt{React}\slashsep
		\texttt{[Type|Java]Script}\slashsep
		\texttt{Gitlab CI/CD}\slashsep
		\texttt{AWS [Amplify|S3]}
	}
\end{entrylist}

\begin{entrylist}
	\entry
	{2019 -- 2020\\\footnotesize{freelance}}
	{Sviluppatore Full Stack}
	{App Android/iOS "Ritm-U"}
	{
		Ho sviluppato un'app mobile con un amico per un cliente. Questa app
		monitora la respirazione diaframmatica tramite il sensore accelerometro
		e analizza il grafico respiratorio ottenuto, se sei curioso puoi vederla su
		\href{https://play.google.com/store/apps/details?id=com.ritmu.app}{Google Play}
		o su \href{https://apps.apple.com/us/app/id1514766559}{App Store}. È stata un'esperienza davvero stimolante
		perché era la prima volta che affrontavo un caso del mondo reale. In particolare
		ho progettato l'architettura, sviluppato tutta la parte back-end,
		sviluppato un algoritmo che analizza la frequenza respiratoria utilizzando tecniche di machine learning e sviluppato alcuni componenti front-end.
		\\
		\texttt{Dart}\slashsep
		\texttt{Python}\slashsep
		\texttt{[Type|Java]Script}\slashsep
		\texttt{NodeJS}\slashsep
		\texttt{Flutter}\slashsep
		\texttt{React}\slashsep
		\texttt{gRPC}\slashsep
		\texttt{graphQL}\slashsep
		\texttt{Docker}
	}
\end{entrylist}


%----------------------------------------------------------------------------------------
%	EDUCATION
%----------------------------------------------------------------------------------------

\cvsect{Istruzione}

\begin{entrylist}
	\entry
		{2018 -- 2023}
		{Laurea Magistrale}
		{Università di Milano}
		{
			Voto finale: 110/110 con lode\\
			Titolo tesi: Un sistema di tipi estensibile per lo sviluppo del protocollo language server\\
			Relatore: \href{https://cazzola.di.unimi.it}{Walter Cazzola}
		}
	\entry
	{Sep 2019 - \\Feb 2020}
	{Erasmus}
	{Università di Paris Saclay}
	{
		Ho seguito un master in Data Science e superato 5 esami.
		È stata un'esperienza davvero intensa dove ho migliorato molto le mie capacità di adattamento.
	}
	\entry
	{2015 -- 2018}
	{Laurea Triennale}
	{Università di Milano}
	{
		Voto finale: 108/110\\
		Titolo tesi: DSL in Neverlang per reti di Petri riflessive\\
		Relatore: \href{https://cazzola.di.unimi.it}{Walter Cazzola}
	}
	\entry
	{2014}
	{Diploma di Maturità}
	{Liceo Classico Nunzio Vaccalluzzo LEONFORTE (EN)}
	{
		Voto finale: 100/100\\
	}
\end{entrylist}

%----------------------------------------------------------------------------------------
%	LANGUAGES & HOBBIES
%----------------------------------------------------------------------------------------

\begin{minipage}[t]{0.5\textwidth}
	\vspace{-\baselineskip} % Required for vertically aligning minipages

	\cvsect{Lingue}

	\textbf{Italiano} - madrelingua\\
	\textbf{Inglese} - Molto buono nella lettura e scrittura. Riesco a capire e parlare ma mi piacerebbe andare in un paese anglofono per perfezionarlo.\\
	\textbf{Francese} - Sono molto bravo in francese grazie alla mia esperienza a Parigi.
\end{minipage}
\hfill
\begin{minipage}[t]{0.3\textwidth}
	\vspace{-\baselineskip} % Required for vertically aligning minipages

	\cvsect{Hobby e Interessi}

	\badge{Sport}
	\badge{Cycling}
	\badge{Movies}
	\badge{Reading}
	\badge{Escursionism}
	\badge{Environmentalist}
	\badge{Tech}
\end{minipage}


%----------------------------------------------------------------------------------------

\end{document}
